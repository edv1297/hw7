\documentclass{article}
\usepackage[utf8]{inputenc}

\title{Static v. Dynamic Scoping}
\author{Eddy Varela & Taylor Beebe}
\date{November 2018}

\begin{document}

\maketitle
\section{Code Snippet}
\begin{verbatim}
1 let result() =
2 let x = 2
3 let f = fun y -> x + y
4 let x = 7
5 x +
6 f x
\end{verbatim}

\subsection{Static Scoping}
The value of x + f x will be 7+(2+7) = 16\newline{}

Line 3: x =2

Line 5: x=7

Line 6: x = 7\newline{}

 In a statically scoped languages, whenever a variable is referenced we look for the last time that variable was referenced in the program (look for an earlier time in the source code). For example in line 3 we reference x+y so this is in reference to x = 2 in line 2. However, in line 5 we reference the assignment to line 4 where x = 7.


\subsection{Dynamic Scoping}
The value of x + f x would be 7 + (7+7) = 21\newline{}

Line3 : x = 7 Because we need to trace when f x was called. When we define f this does not get put on the stack. Thus we look at when f x get called and find the last time x was defined on the stack.\newline{}

Line5 : x = 7
If we follow the trace/look on the call stack, we can see that on the line above, x is now 7. \newline{}

Line6 : x = 7
Finally for the last occurrence of x, we look to the last place where x was defined which was line 4

\subsection{F\#}
Static scoping because F\# uses variable names to hold values. Thus if you define a function and call it f, you can redefine f inside of f and F\# will consider the inner f as a new function.

\end{document}
